\documentclass{article}
\usepackage{graphicx}
\usepackage{xcolor}
\usepackage[hidelinks]{hyperref}
\begin{document}
\begin{center}
SUNY Polytechnic Institute \\
COLLEGE OF BUSINESS\\
DEPARTMENT OF FINANCE AND ACCOUNTING
\\
{\bf Financial Institutions and Markets}\\
{\bf FIN 341}\\
{\bf Spring 2023}\\
\end{center}
\vspace*{5pt}
{\bf Instructor}: Dr. Matthew Brigida\\
{\bf Office}: Donovan 1277\\
{\bf Office Hours}: Tuesday 4pm--6pm \\
{\bf Email}: \href{mailto:matthew.brigida@sunypoly.edu}{\textcolor{blue}{matthew.brigida@sunypoly.edu}}\\
\\
%{\bf Class Location}:  Donovan 1107\\
{\bf Class Location}:  Online D2L/Brightspace\\
%{\bf Class Day \& Time}: Tuesday 6:00--9:30pm \\
{\bf Class Day \& Time}: Asynchronous \\
\\
  {\bf Optional Text}: {\it Financial Markets and Institutions} by Jeff Madura, 10th Edition.\\
  \\
  {\bf Lecture Notes}: \href{https://financial-education.github.io/fin_mkts.html}{\textcolor{blue}{\url{https://financial-education.github.io/fin_mkts.html}}}
\\
\\
\begin{center}
{\bf DESCRIPTION}
\end{center}  
An introductory survey of financial institutions and their respective roles. The
learning outcomes for this course are summarized below:
\begin{enumerate}
\item  Understanding the role of financial markets and institutions in promoting economic
growth. 
\item Introduce the Federal reserve and the U.S. monetary system as well as the reasoning,
method, and effect of open market operations on money and capital markets.
\item Compare and contrast the roles of money, capital (debt and equity), and derivative markets.
\end{enumerate}
\begin{center}
{\bf ACADEMIC HONESTY POLICY}
\end{center} 
Academic dishonesty will not be tolerated in this class. Cheating
on quizzes, examinations, and other forms of dishonesty (e.g., plagiarism, collusion, and
falsification of data) will be dealt with in a serious and formal manner. The penalty for academic
dishonesty in this class will be course failure. That is, any student who is found to be cheating
or engaged in other academically dishonest behavior will be failed for this course for this
semester. Course withdrawals to avoid such a failure will not be permitted. As a student, you
have a responsibility to become familiar with the Academic Honesty Policy found in the {\it Student
 Handbook}.\\
\\
\\
%% \begin{tabular}{|p{2.1 in}|p{3.1 in}|} \hline
%% \multicolumn{2}{|c|}{\bf BSBA Learning Goals and Objectives} \\ \hline
%% {\bf Goal or Objective} & {\bf Assessed by:} \\ \hline
%% Goal 1.0: Demonstrate Business Disciplinary Competence & The exams and homeworks will evaluate a core area of finance: Understanding and operating within financial markets with financial institutions. \\ \hline
%% Goal 3.0 (Objectives 3.1 and 3.2): Communicate Effectively Orally and in Written Form & The class homework will evaluate students' written communication. \\ \hline
%% Goal 4.0 (Objectives 4.1 and 4.3): Demonstrate Analytical Thinking Skills & Students will evaluate differing risk/ return characteristics among asset classes, and decide when it is appropriate to use a particular security. \\ \hline
%% Goal 5.0: Understand Global Issues in the Functional Areas of Business & Understanding contemporary international financial markets and institutions and their purposes. This understanding is evaluated through the exams and homeworks. \\ \hline
%% Goal 6.0 (Objectives 6.1 and 6.3): Demonstrate Effective Use of Technology and Data Analysis & In the homeworks, students will analyze data and communicate conclusions using Excel. \\ \hline
%% \end{tabular}

\begin{center}
{\bf EXAMS} 
\end{center}
There will be three exams (two during the semester and a final exam). No make-up exams will be given.  Failure to take an exam will result in a grade of zero for the missed exam.  
%% \begin{center}
%% {\bf ASSIGNMENTS} 
%% \end{center}
%% There will be three assignments during the semester. The three assignments will be due the week before each exam. Each assignment will be worth 3 and 1/3 final grade points. Late homework will not be accepted.
\begin{center}
{\bf PARTICIPATION} 
\end{center}
% You are expected to attend class and participate in the discussion.
You must make six discussion board posts (on one each topic below).  Be sure to not repeat what others have written---you must add something new to the discussion in your post.  In this sense, it is better to post early.

Use your discretion on how long your post should be. Think of it as sending an email to your boss who asked a particular question---it would be odd to ask how long your email should be.  Moreover, someone can write a very impactful paragraph, and others may say nothing of note in 10 pages. Note, as a general rule you want the following ratio to be as high as possible:

$$\frac{Information\ Conveyed}{Words\ Used}$$

\begin{enumerate}
\item Describe the most pressing challenge for financial institutions today.  For example you could discuss:
\begin{itemize}
\item The effect of Apple enabling saving deposits accounts from the iPhone.
\item Cryptocurrencies and related services.
\item An increasing interest rate environment---the causes of the 2023 regional banking crisis.
\end{itemize}

\item Summarize a recent {\it Wall Street Journal} article on banking.  For example you could explain the arbitrage the Fed afforded banks with the `Bank Term Funding Program'.  You can access the WSJ online for free through the library website.

\item Summarize a famous speculative bubble, and give your opinion whether there are repeating patters across most bubbles.  You may want to look at (all are freely available):
  \begin{itemize}
  \item Manias, Panics, and Crashes by Kindleberger
  \item Kindleberger Cycles: Method in the madness of crowds? \url{https://www.nber.org/system/files/working_papers/w28411/w28411.pdf}
  \item Extraordinary Popular Delusions and the Madness of Crowds, by Mackay
  \item The Minsky Moment  concept: \url{https://en.wikipedia.org/wiki/Minsky_moment}
  \end{itemize}

\item Goldman Sachs Synthetic CDO Case:  State your opinion on whether Goldman did anything wrong (and if so what it was).  What effect do you think the case had on the Synthetic CDO market.  
\item NYSE Case: State your opinion on whether the NYSE did anything wrong (and if so what it was).  What effect do you think the case had on stock markets.
\item What is your opinion of `Life After Debt', and do you think the US will every face this question again?
\end{enumerate}

% \begin{center}
% {\bf PAPERS} 
% \end{center}
% One short paper will be due on the last day of class. You may choose your topic
% from a list of four (or possibly more topics if new ones arise during the semester). Also, if
% you have a particular topic on which you would like to write, relating to financial markets or
% institutions, then you may request this topic from the instructor. The paper should be no more
% than 5 pages double spaced. You will be graded on the clarity, concision, and thoughtfulness
% of your writing, and not on the amount of words you have put on paper. A list of four possible
% topics, as well as source files for these topics, is available in the content section of D2L.
\begin{center}
{\bf PROJECT -- Optional}
\end{center}
Note:  to do an optional assignment you must (1) have a B average so for in the course and (2) discuss your project with me before the midway point of the semester.  You can't wait until the course is nearly over, and then attempt to complete a project.

Students will either:

\begin{itemize}
\item Create a Shiny interactive web application.  To do so you'll first need to sign up for a free \href{https://www.shinyapps.io/}{\textcolor{blue}{shinyapps}} account.  

You are free to create the account under a pseudonym, so no one can publicly identify you as the owner of the account.  However, the web application is a useful tool to show off your work, and is something that can go on your resume (with a link to the application).  So you may prefer to use your real name.  My user name is `mattbrigida'.  

Your application should have something to do with currency markets, and should be at least somewhat original.  See a gallery of applications here:  \href{http://shiny.rstudio.com/}{\textcolor{blue}{shiny.rstudio}}. Possible applications may be:
\begin{itemize}
\item Plot a time series of interest rates.
\item Plot the relationship of mortgage to inflation or interest rates.
\item A mortgage payment calculator.
\item Duration calculator.
\item Create a histogram or probability density plot for bond or stock returns.
\end{itemize}
To get started you will want to use the RStudio development environment for R.  This is available in the Still hall computer lab, or you can install it for free on your own computer from here:  \href{https://www.rstudio.com/products/rstudio/download/}{\textcolor{blue}{download}}.  If you install it on your own computer you'll need to install R first.  You can get R here:  \href{https://cran.r-project.org/}{\textcolor{blue}{download}}

\item Write a paper summarizing a topic relevant to the course.  A set of documents will be posted in which you may be able to find relevant topics.  The paper should be long enough to cover the topic, but should not contain irrelevant filler.
\end{itemize}

\begin{center}
{\bf COURSE COMMUNICATION}
\end{center}
Course Communication: All important/official announcements will either be posted on Blackboard
or emailed to each student's SUNY Poly email account.\\
\\
This is an in-class course, and so you should ask all course related and non-personal questions in class.  If you email a question like ``How do you calculate duration?'' or ``When is the exam?'', my response will be to ask in class.  You may also lose participation points.\\
\\
\begin{center}
\begin{tabular}{lcr}
\multicolumn{3}{c}{\bf GRADING:} \\ \hline
Exam 1 & ................. & 25 \\

Exam 2 & ................. & 25 \\

Final Exam & ................. & 25 \\

% Homework & ................. & 20 \\

% Assignments & ................. &  10\\
Participation & ............... &  25\\
  
% Paper  & ................. &  20 \\

% Project  & ................. &  20 \\

%Paper 2 & ................. & 10 \\

Total Points & ................. & 100\\
\end{tabular}
\end{center}
%%
%%
%% \vspace*{5pt}
%% \begin{center}
%% {Final grades will be assigned according to the following scale}:
%% \end{center}
%% \begin{center}
%% \begin{tabular}{lr}
%% 90 - 100 &  A \\
%% 80 - 89.9 &  B \\
%% 70 - 79.9 &  C \\
%% 60 - 69.9 &  D \\
%% $<$ 60 &  F \\
%% \end{tabular}
%% \end{center}
%%
%%
\vspace*{5pt}
\begin{center}
{\bf An Important Note on Grading}
\end{center}
\begin{enumerate}
\item  There is no special consideration if you need a certain grade in this course to graduate.  {\bf If you require a certain grade in this class to graduate it is your responsibility to earn that grade.} Specifically if you receive a `D' in this course I will not allow you to do extra assignments after the course is complete in exchange for a higher grade.
\item  Late work will not be accepted.  If you do not submit an exam or other assignment by the due date/time you will receive a 0.
\end{enumerate}
%%
%%
\begin{center}
{\bf Adding and dropping this course}
\end{center}
The instructor is not involved in any way with your adding and dropping the course.  It is the student's responsibility to abide by all proper procedures and dates.  \\
%% \begin{center}
%% {\bf GENERAL NOTES}:
%% \end{center}
%% \begin{enumerate}
%% \item Attending class, and reading the text is required.
%% \item All exams will be closed book.
%% \item If you are late for an exam, no extra time will be allotted to you.
%% \item There will be no make up exams or extra points assignments.
%% \item You will be responsible for any material covered in class that is not in your text.
%% \item You should bring your text to class.
%% \item You are expected to be on time for class. This is especially important for exam
%% dates.
%% \item Disruptive behavior in the classroom will not be tolerated.
%% \item You may not use tobacco products in class.
%% \end{enumerate}



%% \begin{center}
%% \vspace*{5pt}
%% {\bf TENTATIVE OUTLINE}
%% \end{center}
%% \begin{itemize}
%% \item Week 1:  Chapter 1
%% \item Week 2:   Chapter 2 
%% \item Week 3:  Chapter 3 
%% \item Week 4:  Chapter 4
%% \item Week 5:  Chapter 5
%% \item Week 6:  Chapter 6
%% \item Week 7:  Exam 1 (February 28)
%% \item Week 8:  Chapter 7
%% \item Week 9:  Chapter 8
%% \item Week 10:  Chapter 9
%% \item Week 11:  Chapter 10
%% \item Week 12:  Exam 2 (April 11)
%% \item Week 13:  Chapter 11
%% \item Week 14:  Exam Review
%% \end{itemize}

\end{document}

